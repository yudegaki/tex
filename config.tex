%platex
%platex -shell-escape 
%dvipdfm
\documentclass[11pt,a4paper]{jarticle}
\usepackage{url}
\usepackage{setspace}
\usepackage{color}
\usepackage[dvipdfmx]{graphicx}
\usepackage{here}
\usepackage{subcaption}
\usepackage{xcolor}
\usepackage{minted}
\renewcommand{\baselinestretch}{0.9}
\renewcommand{\thesection}{\arabic{section}}
\let\oldenumerate\enumerate
\renewcommand{\enumerate}{
\oldenumerate
\setlength{\itemsep}{0pt}
\setlength{\parskip}{0.2pt}
\setlength{\partopsep}{0pt}
\setlength{\topsep}{0pt}
}

%sectionとsubsectionの縦幅設定
\makeatletter
\renewcommand{\section}{\@startsection{section}{1}%
{\z@}%
{1.5ex}%
{1.5ex}%
{\normalfont\Large\bfseries}}
\renewcommand{\subsection}{\@startsection{subsection}{2}%
{\z@}%
{1.5ex}%
{1.5ex}%
{\reset@font\large\bfseries}}
\makeatother

%pdfの幅設定
% ######## measure #########
% # mm = 1mm = 2.85pt      #
% # cm = 10mm = 28.5pt     #
% # in = 25.4mm = 72.27pt  #
% # pt = 0.35mm = 1pt      #
% # em = width of [M]      #
% # ex = height of [x]     #
% # zw = width of [Kanji]  #
% # zh = height of [Kanji] #
% ##########################
% ##################### Portrait Setting #########################
% # TOP = 1inch + \voffset + \topmargin + \headheight + \headsep #
% #     = 1inch + 0pt + 4pt + 20pt + 18pt (default)              #
% # BOTTOM = \paperheight - TOP -\textheight                     #
% ################################################################
\setlength{\textheight}{\paperheight}   % 紙面縦幅を本文領域にする(BOTTOM=-TOP)
\setlength{\topmargin}{-10truemm}       % 上の余白を30mm(=1inch+4.6mm)に
\addtolength{\topmargin}{-\headheight}  % 
\addtolength{\topmargin}{-\headsep}     % ヘッダの分だけ本文領域を移動させる
\addtolength{\textheight}{-35truemm}    % 下の余白も30mm(BOTTOM=-TOPだから+TOP+30mm)
% #################### Landscape Setting #######################
% # LEFT = 1inch + \hoffset + \oddsidemargin (\evensidemargin) #
% #      = 1inch + 0pt + 0pt                                   #
% # RIGHT = \paperwidth - LEFT - \textwidth                    #
% ##############################################################
\setlength{\textwidth}{\paperwidth}    % 紙面横幅を本文領域にする(RIGHT=-LEFT)
\setlength{\oddsidemargin}{-10truemm}  
\setlength{\evensidemargin}{-10truemm}  
\addtolength{\textwidth}{-30truemm}     
\setlength\floatsep{0pt}
\setlength\textfloatsep{0pt}
\setlength\abovecaptionskip{0pt}
\setlength\intextsep{0pt}

\definecolor{bg}{rgb}{0.85,0.85,0.85}